\chapter{はじめに}
\label{chapter:Introduction}

% ページ番号を1にする
\setcounter{page}{1}
\pagenumbering{arabic}

%%%%%%%%%%%%%%%%%%%%%%%%%%%%%%%%%%%%%%%%%%%%%%%%%%%%%%%%%%%%%%%%%%%%%%%%%%%%%%%



近年,金融分野に対して数学や情報科学を活用したファイナンシャルテクノロジーが注目を
集めている.このファイナンシャルテクノロジーの一分野として,資産運用の一つである株取引
に対して,あらかじめ定めておいた手順に従い自動的に注文の数量やタイミングを判断して取
引を実行するアルゴリズム取引が盛んに行われている.また,様々なトレーディングアルゴリズ
ムを過去のビッグデータを用いて検証可能なトレーディングツールも多数提案されている.

%%%%%%%%%%%%%%%%%%%%%%%%%%%%%%%%%%%%%%%%%%%%%%%%%%%%%%%%%%%%%%%%%%%%%%%%%%%%%%%
このトレーディングアルゴリズムに対するこれまでの研究例として,森本の研究\cite{morimoto}がある.森本は該当研究において,
一定期間における安値の最低値を計算し,その変動を折れ線グラフ
で表現したLow グラフと.一定期間における高値の最高値を計算し,そ
の変動を折れ線グラフで表現したHigh グラフに着目し,売買タイミングを
判断するアルゴリズムの考察と提案をしている.該当研究では考察・提案
した複数のアルゴリズムで既存の取引手法である,ゴールデンクロス・デッドクロスを用いたアル
ゴリズムが最終利益,リスクの少なさの点で最も優れることを示している.また,損切りや防衛損切り,
利益確定を行うことで最終利益が向上する傾向があることも示されている.

% 本研究では,過去の株式チャートを使用して,移動平均線乖離率,ゴールデンク
% ロス・デッドクロスという既存アルゴリズムの検証と,新しくLow グラフ・High グ
% ラフを使用するアルゴリズムの提案を行った.また利益最大化手法として,既存の
% 損切り・利益確定の効果検証と,それらを補う「防衛損切り」の提案を行った.実
% 験結果より既存の取引手法である,ゴールデンクロス・デッドクロスを用いたアル
% ゴリズムが最終利益,リスクの少なさの点で最も優れることが判明した.また,ア
% ルゴリズムごとに適切なパラメータは異なるものの,損切りや防衛損切り,利益確
% 定を行うことで最終利益が向上する傾向があることがわかった.
% 今後の課題として,各アルゴリズムにおいてより良い結果が得られるパラメータ
% の調査と,サンプルとなる株式銘柄数を上昇させることが挙げられる.

% 岩永は該当研究において,株式の売買期間を数日から数週間程度の比較
% 的短期間に設定する手法である「スイングトレード」に着目し,売買タイミングを
% 判断するアルゴリズムの考察と提案を行っている.なお,該当研究では考察・提案
% した複数のアルゴリズムのうち,5 日移動平均線と25 日移動平均線による「ゴール
% デンクロス・デッドクロス」のアルゴリズムにおいて最良の結果が得られている.


本研究では,様々なテクニカル指標を組み合わせて有効なトレーディングアルゴリズムの提案を目標とする.また,提案アルゴリズムについては,トレーディングツールを用いて有効性の検証を行う.
用いる指標はMACD,ボリンジャーバンド,60 日移動平均線,日経平均株価の4つであり,MACD とボリンジャーバンドを主体に既存アルゴリズムも含めて 13 通りのアルゴリズムを提案する.

また,提案アルゴリズムは,株価データに対してバックテストを行うためのツールの一つである backtesting.py を用いて実装するとともに,
実際の東証の上位の株価データに対する適用を行い,提案アルゴリズムの有効性の検証を行う.検証結果より,既存の2つのアルゴリズムと,4つの提案したアルゴリズムにより,10年間の期間において100%を超える利益が得られることを示す.また,利益,取引数は低いが平均勝率が約67%となる提案アルゴリズムも示す.

本論文は, 以下のように構築される. 第2章では, Python 言語の概要,本研究で使用した Python 言語のライブラリ,及び, 
株の基本知識と株取引で使われているテクニカル分析の指標ついて説明する.
第3章では, 本研究のテーマであるPython言語を用いたトレーディングアルゴリズムについて述べる. 第4章では, アルゴリズムの実行結果について述べる. 第5章では, 本研究のまとめを行い今後の課題について言及する.

